\documentclass{article}

\usepackage{amsmath}

\begin{document}

\section{Runge-Kutta Methods}
Runge-Kutta methods solve an ODE $\dot{x} = f(x)$, where $x$ is some vector. The idea is to find coefficients $b_i$ and $a_{ij}$ such that 
$$ x_{n+1} = x_n + \sum_{i=1}^s b_i k_i.$$
$k_i$ is a vector of approximate derivatives $k_i\approx f(x_n)$, such that the weighted sum gives a very good approximation to the true answer. The $k_i$ are defined by 
$$ k_i = hf\left( x_n + \sum_{j=1}^s a_{ij}k_j \right).$$
The coefficients of a RK scheme can be put into a Butcher tableau. It is a convention to take $k_1 = f(x_n)$, so there is no top row.
$$\begin{tabular}{c c c}
	$a_{11}$ & $\cdots$ & $a_{1s}$\\
	$a_{21}$ & $\cdots$ & $a_{2s}$\\
	$\vdots$ & $\ddots$ & $\vdots$\\
	$a_{s1}$ & $\cdots$ & $a_{ss}$ \\
	\hline
	$b_1$ & $\cdots$ & $b_s$
\end{tabular}
$$
As an example, suppose we have a two-step method. Expanding both sides of the discrete approximation in time, we get 
$$ x_n + h f(x_n) + \frac12 h^2 f'(x_n) f(x_n) + O(h)^3 = x_n + b_1 k_1 + b_2 k_2.$$
Comparing terms of order $h$ gives 
$$ b_1 + b_2 = 1.$$
Comparing terms of order $h^2$ gives 
$$b_1(a_{11} + a_{12}) + b_2(a_{21} + a_{22}) = \frac12.$$
Here are some solutions.
\subsection*{Huen's Method}
$$\begin{tabular}{c c}
	0 & 0\\
	1 & 0\\
	\hline
	$1/2$ & $1/2$
\end{tabular}
$$
\subsection*{Midpoint method}
$$\begin{tabular}{c c}
	0 & 0\\
	1/2 & 0\\
	\hline
	$0$ & $1$
\end{tabular}
$$
\subsection*{Trapezoid Rule}
$$\begin{tabular}{c c}
	0 & 0\\
	1/2 & 1/2\\
	\hline
	$1/2$ & $1/2$
\end{tabular}
$$

\section{Symplectic Runge-Kutta}
Hamilton's equations of motion conserve a symplectic form $\omega = dq^i \wedge dp_i.$ If an integrator preserves such a form \textit{exactly}, then the integrator is called a symplectic integrator. Let $\omega_n$ be the 2-form at different timesteps. 


\end{document}
